\documentclass[11pt]{article}

\usepackage{sectsty}
\usepackage{graphicx}
\usepackage{amsmath}
\usepackage{algpseudocode}

% Margins
\topmargin=-0.45in
\evensidemargin=0in
\oddsidemargin=0in
\textwidth=6.5in
\textheight=9.0in
\headsep=0.25in

\title{ Neutral Atom Solver}
\author{ Arturo Hernandez Zeledon }
\date{1 November 2023}

\begin{document}
\maketitle	
\pagebreak

% Optional TOC
% \tableofcontents
% \pagebreak

%--Paper--

\section{Units}
This code uses Hartree units where $\frac{\hbar^2}{m_e} = e = 4\pi\epsilon_0 = 1$. $m_e$
is the mass of the electron

\section{Grids}

\subsection{Uniform Grid}


\subsection{Exponential Grid}


\pagebreak
\section{Second Order Differential Equation Solver}
This section focus on solving two differential equations: Schrödinger's equation for one dimensional
systems (including radial potentials for DFT atomic calculations), and 
Poisson's equation, we are then interested on numerically solving equations of the form:
 be written as:
 \begin{equation}
    \label{eqn:general_second_order_diff}
    \frac{d^2y}{dx^2} = f(x)y + g(x)
 \end{equation}

The first step to numerically solving this equation is by rewriting it as a system of linear differential 
equations such that $y(x) \rightarrow y^0(x),$ $\frac{dy(x)}{dx} \rightarrow y^1(x)$ such that equations 
\ref*{eqn:general_second_order_diff} became:
\begin{equation}
    \begin{cases}
        \label{eqn:system_linear_diff}
        \frac{dy^0}{dx} = y^1(x) \\
        \frac{dy^1}{dx} = f(x)y^0(x) + g(x)
    \end{cases}
\end{equation}
This system of equations can be solved with method like Runge-Kutta order 4 and/or predictor corrector
Adams-Moulton order 4.\\
To numerically solve the equations \ref*{eqn:system_linear_diff} the system must be translated into 
 discrete sets of values. First by discretizing the space, into a grid of $N$ values such that $x_i$ is the 
 value of the grid at position "i". The functions are also discrete $f_i \rightarrow f(x_i)$ stands by evaluating
 a function $f$ on the "i" position of the grid. And "$h$" stands for the delta between to consecutive points on the grid
 $h = x_{i+1} - x_{i}$, the grid points are not necessarily uniformly distributed. The objective is to find the $N$ values of
 $y$ over the grid, for a numerical solution the values of $y(x_1)$ and $y(x_2)$ must be known.
 \subsection*{Runge-Kutta order 4}
This subsection describes how a function (RK4) utilizes the Runge-Kutta order 4 method to produce the values $y^0(x_{i+1})$, and 
$y^1(x_{i+1})$.
%
%taking as inputs;  $h_{in} = x_{i+1} - x_{i}$, $f[i : i+1] \rightarrow [f(x_i), f(x_{i+1})] \rightarrow \vec{f}$, 
% $g[i : i+1] \rightarrow [g(x_i), g(x_{i+1})] \rightarrow \vec{g}$, and $y^0(x_{i})$, $y^1(x_{i})$:
 \begin{algorithmic}
    \State $h \gets (x_{i+1} - x_{i})$
    \State $ y^0 \gets y^0(x_{i})$
    \State $ y^1 \gets y^1(x_{i})$
    \State $\bar{f} \gets [f(x_i), f(x_{i+1})]$
    \State $\bar{g} \gets [g(x_i), g(x_{i+1})]$
    \State $k_{01} = h*y^1$
    \State $k_{11} = h*(\bar{f}[1]*y^0 + \bar{g}[1])$
    \State $k_{02} = h*(y^1 + 0.5*k_{11})$
    \State $k_{12} = h[*0.5*(\bar{f}[1] + \bar{f}[2])*(y^0 + 0.5*k_{01}) + 0.5*(\bar{g}[1] + \bar{g}[2])]$
    \State $k_{03} = h*(y^1 + 0.5*k_{12})$
    \State $k_{13} = h[*0.5*(\bar{f}[1] + \bar{f}[2])*(y^0 + 0.5*k_{02}) + 0.5*(\bar{g}[1] + \bar{g}[2])]$
    \State $k_{04} = h*(y^1 + k_{13})$
    \State $k_{14} = h[\bar{f}[2]*(y^0 + k_{03}) + \bar{g}[2]]$
    \State $y^0(x_{i+1}) = y^0 + \frac{1}{6}*(k_{01} + 2*k_{02} + 2*k_{03} + k_{04})$
    \State $y^1(x_{i+1}) = y^1 + \frac{1}{6}*(k_{11} + 2*k_{12} + 2*k_{13} + k_{14})$
    \State \Return $[y^0(x_{i+1}), y^1(x_{i+1})]$
\end{algorithmic}

\subsection*{Predictor Corrector Method Adams-Moulton Orders 4 and 5}
Description of the predictor corrector Adasm-Moulton (PCAM4) function implementation, the PCAM4 the integration routine
produces $y^0(x_{i}), y^1(x_{i})$
\begin{algorithmic}
    \State $h \gets (x_{i} - x_{i-1})$
    \State $ \bar{y}^0 \gets [y^0(x_{i-4}), y^0(x_{i-3}), y^0(x_{i-2}), y^0(x_{i-1})]$
    \State $ \bar{y}^1 \gets [y^1(x_{i-4}), y^1(x_{i-3}), y^1(x_{i-2}), y^1(x_{i-1})]$
    \State $\bar{f} \gets [f(x_{i-3}), f(x_{i-2}), f(x_{i-1}), f(x_{i})]$
    \State $\bar{g} \gets [g(x_{i-3}), g(x_{i-2}), g(x_{i-1}), g(x_{i})]$
    \State $y^0_{prediction} = \bar{y}^0[4] + \frac{h}{24}*(55*\bar{y}^1[4] - 59*\bar{y}^1[3] 
    +37*\bar{y}^1[2] - 9*\bar{y}^1[1])$
    \State $y^1_{prediction} = \bar{y}^1[4] + \frac{h}{24}*(55*(\bar{y}^0[4]*f[4] + g[4]) - 59*(\bar{y}^0[3]*f[3] + g[3]) 
    +37*(\bar{y}^0[2]*f[2] + g[2]) - 9*(\bar{y}^0[1]*f[1] + g[1]))$

    \State $y^0_{corrector} = \bar{y}^0[4] + \frac{h}{24}*(9*y^1_{prediction} + 19*\bar{y}^1[3] 
    - 5*\bar{y}^1[2] + 1*\bar{y}^1[1])$
    \State $y^1_{corrector} = \bar{y}^1[4] + \frac{h}{24}*(9*(y^0_{prediction}*f[4] + g[4]) + 19*(\bar{y}^0[3]*f[3] + g[3]) 
    - 5*(\bar{y}^0[2]*f[2] + g[2]) + 1*(\bar{y}^0[1]*f[1] + g[1]))$
    
\end{algorithmic}

\subsection{Schrödinger's equation}
The time independent Schrödinger equation:
\begin{equation}
    \label{eqn:general_schdinger}
    \frac{-1}{2} \nabla^2 \Psi(\vec{r})  + V(r) \Psi(\vec{r}) = E \Psi(\vec{r})
\end{equation}

Assuming a solution of the form:
\begin{equation}
    \label{eqn:psi_deffinition}
    \Psi(\vec{r}) = \frac{u(r) Y_{m}^{l}}{r}
\end{equation}
Where $Y_{m}^{l}$ are the spherical harmonics such that after substituting equation \ref{eqn:psi_deffinition} 
into equation \ref*{eqn:general_schdinger}, we get a radial Schrödinger equation of the form:
\begin{equation}
    \label{eqn:radial_schdinger}
    \frac{-1}{2} \frac{d^2 u(r)}{dr^2}  + \frac{l(l+1) u(r)}{2r^2} + V(r) u(r) = E u(r)
\end{equation}
Now introduce an effective potential that contains external potential, exchange, correlation, hartree, 
and angular
\begin{equation}
    \label{eqn:V_eff}
    V_{effe}= V_{angu}(l,r) + V_{ext}(*paramters, r) + V_{hart}(\rho, r) + V_{exch}(\rho, r) 
    + V_{corr}(\rho, r)
\end{equation}

Rearranging terms equation \ref*{eqn:radial_schdinger} became:
\begin{equation}
\label{eqn:final_schdinger}
    \frac{d^2 u(r)}{dr^2} = 2 (V_{effe} - E)  u(r)
\end{equation}
And equation \ref*{eqn:final_schdinger} has the form of equation \ref*{eqn:general_second_order_diff}
with $f = 2 (V_{effe} - E)$, $g = 0$, and $y = u$
%--/Paper--


\end{document}